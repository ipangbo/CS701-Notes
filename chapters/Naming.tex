\chapter{Naming}

从我们已知的信息源(KB, 知识库)中提取信息并存入短期记忆(MM)比从外部参考中获取新知识更为高效。这意味着对于那些我们已经知道的概念,我们更容易理解和记忆。访问短期记忆(MM)比访问知识库(KB)更高效。这意味着,对于我们刚刚遇到或刚刚学习的信息,我们能够更快速地回想和处理。最后一个假设指出,访问知识库(KB)比访问外部参考(ER)更为高效。对于已知信息的回忆和处理速度比新知识的获取要快。

如果一个名称能够让我们更好地记住它代表的是什么,那么这个名称很可能会被更好地记录在短期记忆中,并且通常需要更少地参考知识库或外部参考。
\section{使用有意义的名称}
\section{避免编码}
编码指的是按某种特定方式选择名称,以编码各种信息。这通常是为了使代码更易读或理解,但很多时候,它可能导致混淆或过度复杂。

\subsection{编码的例子}
变量命名的前缀:例如,一个将要保存整数的变量应该以i, j, k开头,但不应该以l开头。这种编码方式在某些编程风格中是常见的,但在现代编程实践中,这种风格已经过时,因为它可能导致混淆。
字段名称前缀:所有字段名称都应该以f开头。这是另一种编码示例,但同样,这种编码方式可能会使代码变得难以阅读。
颜色编码:所有字段应该被染成特定的蓝色阴影。但这提出了一个问题:应该选择哪种蓝色?这种方法在视觉上可能有帮助,但如果没有明确的颜色代码,可能会导致混淆。
匈牙利表示法(Hungarian Notation):这是一个特定的编码例子,它在Microsoft Windows的C API中被使用,因为在那时基本上没有类型检查。例如,"arru8NumberList" 这个变量名编码了该变量是一个8位无符号整数数组。

\subsection{编码的问题}
虽然编码的初衷是为了提高代码的可读性,但实际上它可能会引起混淆,尤其是在没有明确规则或者团队成员对规则不熟悉的情况下。
编码可能导致不必要的复杂性,使代码更难维护。

\section{总结}
选择易于添加到 MM 中的名称,或易于在知识库中找到的名称。