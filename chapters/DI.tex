\chapter{依赖注入}

\section{一支穿云箭,千军万马来相见}

依赖注入的核心目标是“将对象的使用与对象的构建分离”。这意味着,在使用对象时,不需要关心如何创建和初始化这个对象。相反,这些对象(或依赖项)会在运行时被注入。

例如,MovieLister是一个示例类。它在编译时并不知道将使用哪个类来为其提供功能,只有在运行时才解决这一依赖。这是动态依赖解析的一个例子,其中具体的实现或服务是在程序运行时确定的,而不是在编写代码时。

\paragraph{对象创建的独立性}
“一个类如何独立于其所需对象的创建?”传统的方式是类内部直接实例化其依赖。但依赖注入提供了一种方法,允许这些依赖从外部被注入,从而确保类与其依赖对象的具体创建和管理方式解耦。
\paragraph{配置的分离}
“如何在单独的配置文件中指定对象的创建方式?”这突出了依赖注入的一个重要方面,即配置的外部化。通过将对象创建的配置放在代码之外(例如,在XML或JSON配置文件中),我们可以更灵活地更改依赖,而无需修改代码。这也支持了应用程序的模块化和可扩展性。

\section{依赖注入的角色}
\paragraph{服务对象(Service)}是将被使用的实际服务或组件。我们可以看到CSVMovieFinder作为一个服务对象示例,它可能提供了一种从CSV文件中查找电影的方法。
\paragraph{客户端对象(Client)}是使用服务的对象。它依赖于一个或多个服务来执行其功能。MovieLister是客户端对象,它需要MovieFinder服务来获取电影列表。
\paragraph{接口(Interface)}定义了服务对象应如何被使用,确保客户端对象与服务对象之间的通信是一致和预期的。在图中,MovieFinder是一个接口,它定义了查找电影的方法或约定,而具体的实现(如CSVMovieFinder)则提供了这些方法的具体逻辑。
\paragraph{注入器(Injector)}是负责创建服务对象并将其注入到客户端对象中的组件。在这个幻灯片中,注入器的具体示例没有明确表示,但它的作用是确保MovieLister获得了它所需要的MovieFinder服务实例。

\section{依赖注入的形式}

\subsection{构造函数注入}
这种注入方式是通过对象的构造函数将依赖传递给对象。在对象被实例化时,所有的依赖都会首先被构造。这确保了对象在使用前已经有了它所需的所有依赖。但是缺乏灵活性。一旦对象被创建,其依赖就不能再改变。这对于需要在运行时更改依赖的情况不是很方便。
\subsection{Setter注入}
这种注入方式是通过对象的setter方法来传递依赖。它具有灵活性。可以在对象的生命周期的任何时候更改其依赖,这允许在运行时更改实现。但由于依赖是通过setter方法注入的,客户端可能会面临一个问题,即如何确保在使用对象之前已经注入了所有必要的依赖。如果缺少某个依赖,可能会导致运行时错误。
\subsection{接口注入}
这种注入方式要求客户端实现一个特定的接口,该接口包含用于注入依赖的方法。它通过接口明确地定义了如何注入依赖,这使得注入器知道如何与客户端通信和传递依赖。但这要求客户端必须实现注入接口,这可能会对客户端造成额外的编程负担。

每种依赖注入的形式都有其优缺点,适用于不同的情境。选择哪种形式取决于具体的需求和上下文。例如,对于需要在对象生命周期中更改其依赖的应用,setter注入可能更为合适;而对于希望确保所有依赖在使用前都已经准备好的应用,构造函数注入可能是更好的选择。

\section{对依赖注入的评价}

\subsubsection{优点}
\paragraph{松耦合 \& 允许任何实现}
依赖注入允许系统的组件独立于其具体的实现。这意味着你可以很容易地替换一个组件的实现而不影响到使用它的其他组件。
\paragraph{依赖于抽象,而不是具体实现}
客户端代码不再需要知道具体的实现或初始化其依赖项。它只需要依赖于一个抽象(例如接口或抽象类),这提高了代码的解耦。
\paragraph{更容易进行单元测试}
由于依赖是从外部注入的,所以在单元测试中可以轻松地使用模拟对象(mocks)来模拟服务,从而使测试更为简单和可控。

\subsubsection{缺点}
\paragraph{额外的接口}
为了实现依赖注入,可能需要创建额外的接口。这可能会导致代码库增大,同时为开发者带来额外的工作量。
\paragraph{额外的复杂性}
尽管DI可以提高代码的模块化,但同时也可能增加代码的复杂性。特别是对于不熟悉DI的开发者来说,他们可能会发现代码难以追踪和理解。
\paragraph{不能在运行时更改依赖(构造函数注入)}
如前所述,使用构造函数注入的对象在创建后不能更改其依赖。这限制了它的灵活性,尤其是在需要在运行时更改依赖的情境中。
\paragraph{代码可能变得难以追踪}
当使用DI时,特别是在大型项目中,由于依赖是在运行时注入的,这可能使得代码的流程变得不那么直观。
\paragraph{注入null作为依赖时会发生什么?}
如果尝试将null作为依赖注入,这可能导致运行时错误或不可预测的行为。应该确保注入的依赖是有效的并且非null。

\section{依赖注入与设计准则(Design Principles)}这个原则的核心思想是一个类应该只有一个改变的原因。在依赖注入的上下文中,单一职责原则指的是将对象的使用与其构造分开。这确保了类的职责清晰分隔,提高了代码的模块化和可维护性。
\paragraph{单一职责原则 (Single Responsibility)}依赖反转原则强调要针对接口编程,而不是针对实现编程。这意味着高级模块不应该依赖于低级模块,而是应该依赖于抽象。这确保了系统的各个部分之间的解耦,使其更易于改变和维护。
\paragraph{依赖反转原则 (Dependency Inversion)}依赖反转原则强调要针对接口编程,而不是针对实现编程。这意味着高级模块不应该依赖于低级模块,而是应该依赖于抽象。这确保了系统的各个部分之间的解耦,使其更易于改变和维护。
\paragraph{控制反转原则(Inversion of Control, IoC)}控制反转是一种设计思想,其中对象不自己创建其依赖对象,而是通过某种外部机制(如依赖注入)来接收它们。依赖注入是控制反转的一种实现形式,其中依赖是从外部注入到对象中的。

\section{依赖注入与设计模式}
\paragraph{工厂模式 (Factory Pattern)}这是一个创建型设计模式,其核心思想是提供一个接口来创建对象,但具体决定实例化哪个类的责任交给子类。在与依赖注入相关的上下文中,工厂模式可以用来构建和提供依赖,而不必在客户端代码中直接实例化它们。
\paragraph{策略模式 (Strategy Pattern)}
策略模式定义了一系列算法,并将每一个算法封装起来,使它们可以互相替换。这种模式的关键在于它允许算法的变化独立于使用算法的客户端。在依赖注入的背景下,策略模式可以用来定义和注入可互换的策略或行为。

\section{依赖注入与可变性}

在添加新特性时,例如使用另一种类型的MovieFinder,需要更改多少现有的类?

当我们想要增加新功能时,一个核心的设计目标是最小化需要更改的类的数量。更少的改动意味着系统更具有可维护性和稳定性。为了实现新的功能,而不更改现有的代码,可以创建一个新的MovieFinder子类。这种做法遵循了开放封闭原则,即软件实体(如类、模块、函数等)应该对扩展开放,对修改封闭。这意味着我们可以添加新功能,而无需更改现有的代码。为了使新的MovieFinder子类能够被正确地使用,我们需要更改负责构建服务并将它们注入到客户端的注入器。这样,注入器就可以创建新的MovieFinder实例并将其注入到需要它的对象中。

$C = 1$。
\section{总结}
依赖注入(DI)是一种设计模式,它涉及到向对象提供其依赖关系,而不是由对象自己创建这些依赖关系。通过这种方式,对象与它们所依赖的具体实现被分离,从而允许更容易的配置、更灵活的系统结构以及更易于测试的代码。

通过使用依赖注入,系统的各部分可以更加灵活地进行修改和扩展。当需要添加新功能或更改现有功能时,可以最小化对现有代码的更改,从而提高软件的可维护性和稳定性。这也体现了现代软件设计的一些核心原则和模式,如单一职责原则、开放封闭原则和依赖反转原则。

DI的核心好处之一是支持松耦合。在没有DI的系统中,对象通常直接创建或查找其依赖项,这导致了高度的耦合。松耦合代码意味着组件间的依赖关系降到最低,从而使得代码更加模块化,每个模块都可以独立地开发、测试和维护。这样的代码更容易适应变化、扩展和维护。

DI强调关注点的分离。这意味着每个对象都只关心其核心业务逻辑,而不必关心如何获得其依赖关系。通过将依赖关系的创建和绑定移交给外部实体(如注入器),每个对象可以专注于其核心职责,从而使得代码更加清晰和易于管理。

如之前讨论的那样,DI有多种形式,包括构造器注入、setter注入和接口注入。每种方法都有其特点和适用的场景,但也带来了不同的权衡。例如,构造器注入在对象创建时确保了所有的依赖都被满足,但它在运行时缺乏灵活性;而setter注入提供了更大的灵活性,但可能使得依赖在使用之前未被正确设置。理解这些权衡有助于在特定的应用场景中选择最适合的DI方法。

依赖注入是一种强大的设计工具,它支持松耦合、关注点分离,并提供了不同的实现方式以满足不同的需求和约束。使用DI可以使软件设计更加模块化、灵活和易于维护。
