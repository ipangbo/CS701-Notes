\section{Potential Assessment Questions}
\subsection{3.2}
Martin’s heuristic for good names Use Intention-Revealing Names says that the name should answer the big questions, including “how the variable, function, or class is used.”
Explain how answering this question improves comprehensibility in terms of the program comprehension model.

The program comprehension model essentially deals with how programmers understand existing code – be it written by someone else or even by the same developer but some time ago. Understanding how code works is essential for debugging, enhancing, or reusing it. The model implies a cognitive process, where a developer constructs a mental model (or representation) of the code by combining the explicit information given in the code with their prior knowledge (often termed as Knowledge Base or KB).

Let's analyze Martin's heuristic for good names, particularly "Use Intention-Revealing Names", in the context of the program comprehension model:

Explicit Information: When names are intention-revealing, they serve as explicit information about what that piece of code (variable, function, or class) is supposed to do. A good, descriptive name reduces the ambiguity or uncertainty about its purpose.

Reduced Dependency on External Resources (ER): If a name doesn't convey its intention, developers might need to consult external resources like documentation, or other parts of the code to understand its purpose. In the comprehension model, accessing ER is less efficient than accessing the Knowledge Base (KB) or Mental Model (MM). Intention-revealing names reduce the need to access ER.

Efficient Interaction with Knowledge Base (KB): Developers come with a set of prior knowledge about naming conventions, design patterns, programming paradigms, etc. When they encounter an intention-revealing name, it resonates with their existing knowledge, making the comprehension process smoother and quicker.

Building an Accurate Mental Model (MM): For effective debugging and modification, developers should build a correct and comprehensive mental model of the code. Descriptive names aid in constructing this model by providing clear and unambiguous cues about how each component fits into the broader picture.

Reduced Cognitive Load: The human working memory, often likened to the MM, has a limited capacity. If developers spend significant mental resources trying to decipher the purpose of poorly named variables or functions, it detracts from their ability to understand the broader logic or flow of the program. Intention-revealing names streamline the comprehension process by reducing this cognitive overhead.

In conclusion, using intention-revealing names is a heuristic that aligns well with the program comprehension model. By providing clear, unambiguous cues about the purpose and usage of code components, these names facilitate a more efficient and accurate understanding of the code, leading to better maintainability and fewer errors.

马丁的 "好名字启发式"(heuristic for good names Use Intention-Revealing Names)认为,名字应能回答重大问题,包括 "变量、函数或类是如何使用的"。
从程序理解模型的角度解释回答这个问题如何提高可理解性。

程序理解模型主要涉及程序员如何理解现有代码--无论是别人编写的代码,还是同一开发人员在很久以前编写的代码。理解代码的工作原理对于调试、增强或重用代码至关重要。该模型意味着一个认知过程,开发人员通过将代码中提供的显式信息与其先前的知识(通常称为知识库或 KB)相结合,构建代码的心智模型(或表征)。

让我们在程序理解模型的背景下分析一下马丁关于好名字的启发式,尤其是 "使用能反映意图的名字":

显式信息: 当名称能够揭示意图时,它们就成为了一段代码(变量、函数或类)应该做什么的明确信息。一个好的、具有描述性的名称可以减少其目的的模糊性或不确定性。

减少对外部资源(ER)的依赖: 如果名称不能表达其意图,开发人员可能需要查阅文档等外部资源或代码的其他部分,才能理解其目的。在理解模型中,访问 ER 的效率低于访问知识库(KB)或心智模型(MM)。揭示意图的名称可减少访问 ER 的需要。

与知识库(KB)的高效交互: 开发人员拥有一套关于命名约定、设计模式、编程范例等方面的先验知识。当他们遇到揭示意图的名称时,就会与现有知识产生共鸣,从而使理解过程更加顺畅和快速。

建立准确的心理模型(MM): 为了进行有效的调试和修改,开发人员应为代码建立一个正确而全面的心智模型。描述性名称可提供清晰明确的提示,说明每个组件是如何融入大局的,从而有助于构建这一模型。

减少认知负荷:人类的工作记忆通常被比作 MM,其容量是有限的。如果开发人员花费大量的脑力去解读命名不清的变量或函数的目的,就会影响他们理解更广泛的程序逻辑或流程的能力。揭示意图的名称可以减少这种认知开销,从而简化理解过程。

总之,使用揭示意图的名称是一种启发式方法,与程序理解模型十分吻合。通过对代码组件的目的和用法提供清晰明确的提示,这些名称有助于更有效、更准确地理解代码,从而提高可维护性,减少错误。

\subsection{4.1}
Explain, in terms of the program comprehension model, why we would generally expect names created by joining whole words together would lead to more comprehensible code than if we used names that are formed by taking the first letters of the same words (acronyms).

The program comprehension model describes the cognitive process developers undergo when trying to understand a piece of code. This involves creating a mental model (MM) based on the code itself and their prior knowledge (Knowledge Base, or KB). When they can't figure something out based on these two sources, they often resort to External Resources (ER) such as documentation or other reference materials.

Explicitness and Ambiguity: Names created by joining whole words are more explicit and convey more information than acronyms, which are often ambiguous. For a developer reading the code, whole words are easier to understand without referring back to prior context or definitions, thereby reducing cognitive load.

Knowledge Base (KB) Access: Whole words often resonate better with a developer's existing knowledge. Acronyms might not immediately connect with the KB unless the developer is familiar with the specific context where the acronym was created.

Reduced Dependency on External Resources (ER): When encountering an unfamiliar acronym, a developer might need to resort to external documentation or search for the place where the acronym was first defined, leading to inefficiencies. Whole-word names often make the code self-explanatory, reducing the need to refer to ER.

Building an Accurate Mental Model (MM): Whole words contribute to a clearer and more accurate mental model, as they provide more context. Acronyms, being condensed and potentially ambiguous, might lead to misconceptions or gaps in the MM.

从程序理解模型的角度解释,为什么我们通常会认为,与使用由相同单词的首字母组成的名称(缩略语)相比,由整个单词连接而成的名称更容易理解代码。

程序理解模型描述了开发人员在试图理解一段代码时所经历的认知过程。这包括根据代码本身和他们先前的知识(知识库或 KB)创建一个心智模型(MM)。当他们无法根据这两个来源搞清楚某件事情时,他们通常会求助于外部资源(ER),如文档或其他参考资料。

明确性和模糊性: 通过连接整词创建的名称比缩略语更明确,传递的信息也更多,而缩略语往往模棱两可。对于阅读代码的开发人员来说,整词更容易理解,无需参考先前的上下文或定义,从而减轻了认知负担。

知识库(KB)访问: 整词通常能更好地与开发人员的现有知识产生共鸣。缩略词可能无法立即与知识库产生共鸣,除非开发人员熟悉缩略词产生的具体语境。

减少对外部资源(ER)的依赖: 遇到不熟悉的首字母缩略词时,开发人员可能需要借助外部文档或搜索首次定义该首字母缩略词的地方,从而导致效率低下。整词名称通常会使代码变得不言自明,从而减少参考资源的需要。

建立准确的心智模型(MM): 全词提供了更多的上下文,有助于建立更清晰、更准确的心智模型。首字母缩略词过于简洁,可能会产生歧义,从而导致心智模式中出现误解或空白。

\subsection{4.1}
Give an example where using an acronym for a name is likely to produce more comprehensible code than if the full words were used.

Consider a widely recognized and accepted acronym like "HTTP" (Hypertext Transfer Protocol). In the context of web development or networking:

Using HTTPRequest is likely to be more comprehensible than HypertextTransferProtocolRequest because:

Familiarity: For developers in this domain, "HTTP" is a familiar and instantly recognizable acronym.

Brevity: Shorter names are quicker to read and reduce visual clutter in the code, especially when used frequently.

Industry Standard: Since "HTTP" is an industry-standard term, using the full form might even seem unconventional and counterintuitive.

Thus, while whole-word names generally enhance comprehensibility by providing clarity, there are situations where well-established acronyms can be more effective.

请举例说明使用缩写名称可能比使用全称产生更易理解的代码。

考虑一下像 "HTTP"(超文本传输协议)这样被广泛认可和接受的缩写。在网络开发或网络连接中:

使用 HTTPRequest 可能比使用 HypertextTransferProtocolRequest 更容易理解,因为:

熟悉: 对于该领域的开发人员来说,"HTTP "是一个熟悉的缩写,一眼就能辨认出来。

简洁: 较短的名称阅读起来更快,并能减少代码中的视觉干扰,尤其是在频繁使用时。

行业标准: 由于 "HTTP "是一个行业标准术语,使用全称可能会显得不合常规,有违直觉。

因此,虽然全词名称通常能通过提供清晰度来提高可理解性,但在某些情况下,成熟的缩写可能会更有效。

\subsection{4.2}
For the comprehension model discussed in the course, explain the difference between the mental model and the knowledge base. Use (small!) code examples to illustrate the difference.

The comprehension model often referred to when discussing software comprehension, posits that readers of code build an internal representation (or "mental model") of the code, influenced by their prior knowledge and experiences (the "knowledge base"). Let's break down these two concepts:

Mental Model:

The mental model is a cognitive representation of the system or code. When a developer reads code, they form an understanding of what that code does, how it works, and its purpose. This understanding, built progressively, is the mental model.

It's dynamic and evolves as a developer reads more code or gains more context.

The mental model isn't always correct; misconceptions or misunderstandings can be introduced based on how the code is written or the developer's own biases.

Example:

def add(a, b):
return a + b

Upon reading this code, a developer might form the mental model: "This is a function that takes two parameters and returns their sum."

Knowledge Base:
The knowledge base comprises all the prior knowledge, experiences, and concepts that a developer brings to the table when reading code. This includes programming constructs, patterns, algorithms, domain-specific knowledge, and even prior experiences with similar code.

It's static in the sense that, at any given moment, it represents what the developer knows up to that point. However, it's constantly expanding and evolving over time as the developer learns and experiences more.

Example:

def fibonacci(n):
if n <= 1:
return n
else:
return fibonacci(n-1) + fibonacci(n-2)

A developer with prior knowledge about the Fibonacci sequence (from their knowledge base) will immediately recognize this recursive pattern and comprehend the code's intention. In contrast, a developer without that prior knowledge might need more time to build their mental model around how this function works.

针对课程中讨论的理解模型,解释心智模型和知识库之间的区别。使用(小型!)代码示例来说明两者的区别。

在讨论软件理解时经常提到的理解模型认为,代码读者受其先前知识和经验("知识库")的影响,建立了代码的内部表征(或 "心智模型")。让我们来分解这两个概念:

心智模型:

心智模型是对系统或代码的认知表征。当开发人员阅读代码时,他们会对代码的作用、工作方式和目的形成一种理解。这种逐步形成的理解就是心智模型。

它是动态的,随着开发人员阅读更多代码或获得更多上下文而不断发展。

心智模型并不总是正确的;基于代码的编写方式或开发人员自身的偏见,可能会产生误解或误解。

代码略。

在阅读这段代码时,开发人员可能会形成这样的心理模型: "这是一个接收两个参数并返回其和的函数。

知识库:
知识库包括开发人员在阅读代码时所掌握的所有先前知识、经验和概念。其中包括编程结构、模式、算法、特定领域的知识,甚至是以前使用类似代码的经验。

它是静态的,因为在任何给定的时刻,它都代表着开发人员到那时为止所掌握的知识。但是,随着时间的推移,随着开发人员学习和体验的增加,它也在不断扩展和发展。

代码略。

如果开发人员事先了解斐波那契数列的相关知识(来自他们的知识库),就会立即识别出这种递归模式,并理解代码的意图。相比之下,不具备相关知识的开发人员可能需要更多时间来围绕该函数的工作原理建立心智模型。

\subsection{5.1}
Consider the following code fragment (which comes from the JUnit 4.11 implementation for org.junit.Assert):

/**
* Protect constructor since it is a static only class
*/
protected Assert() {
    }

Briefly comment on the usefulness of this comment.

This is a constructor for the Assert class, which is declared as protected. this constructor is empty and does not perform any operations.

The comment that says "Protect constructor since it is a static only class" is meant to protect the constructor since it belongs to a class that has only static methods.

Now, let's relate this code to the Program Comprehension Model (PCM):

MM (Mental Model): The code reader needs to form a mental model of the code to understand how it works. Without annotations, the reader may have questions about why the constructor is empty, why it is protected, and so on.

KB (Knowledge Base): The reader's prior knowledge may tell them that when a class has only static methods, it is often prevented from instantiating it. But this is not known to everyone, especially developers who are not familiar with Java or OOP.

ER (External Representation): The code and comments themselves constitute the external representation. The comments provide additional context here, explaining why the constructor is protected.

Assimilation Process: This is the process of merging ER into MM and KB. Annotation helps this process as it provides the reader with the reason why the constructor is protected.
Effectiveness and Efficiency.

Effect: With this comment, the reader of the code can more quickly understand why the constructor is empty and why it is protected. This makes the code more self explanatory.
Efficiency: Without the annotation, readers may need more time and effort to surmise why the constructor is protected. But with this comment, they can get the answer immediately, improving the efficiency of understanding the code.

请看下面的代码片段(来自 JUnit 4.11 的 org.junit.Assert 实现):

代码略

请简要说明该注释的用处。

这是一个Assert类的构造函数,它被声明为protected。此构造函数是空的,不执行任何操作。

注释指出:“Protect constructor since it is a static only class”,意思是为了保护该构造函数,因为它属于一个只有静态方法的类。

现在,我们将这段代码与程序理解模型(Program Comprehension Model)进行关联:

MM (Mental Model): 代码阅读者需要形成一个关于代码的精神模型,理解它是怎样工作的。没有注释,阅读者可能会对为什么构造函数是空的、为什么它是受保护的等有疑问。
KB (Knowledge Base): 阅读者的先前知识可能告诉他们,当类只有静态方法时,常常会阻止实例化它。但这并不是所有人都知道的,特别是那些不太熟悉Java或OOP的开发者。
ER (External Representation): 代码和注释本身构成了外部表示。注释在这里提供了额外的上下文,解释了为什么构造函数是受保护的。
Assimilation Process: 这是将ER合并到MM和KB中的过程。注释帮助了这一过程,因为它为阅读者提供了为什么构造函数受保护的原因。
效果与效率:

效果:有了这个注释,代码的阅读者可以更快地理解为什么构造函数是空的和为什么它是受保护的。这使得代码更具有自解释性。
效率:在没有注释的情况下,阅读者可能需要更多的时间和努力来推测为什么构造函数是受保护的。但有了这个注释,他们可以立即得到答案,提高了理解代码的效率。

\subsection{6.1}
Suppose you were shown two designs for a system, Design A, which has 5 classes, and Design B, which has 9 classes. For both designs, you agree that the classes are reasonable. Which design do you think is likely to be more comprehensible and why?

Number of Classes: Although Design B has more classes, it may be easier to understand if these classes make the responsibilities of each class clearer and reduce the complexity between classes. In contrast, Design A may have only five classes, but if the interactions between them are very complex, it may be difficult to understand.

Effectiveness: If each class in Design B has a clear, single responsibility and clear interactions with other classes, it may be easier for developers to form an accurate mental model, even if there are more classes.

Efficiency: If the five classes in Design A are relatively functionally focused, it may take more time and effort to understand the internal details of each class. In Design B, on the other hand, each class may be smaller and more focussed, making understanding each class a quicker and more straightforward task.

Conclusion: It is not possible to conclude which design is easier to understand from the number of classes alone. Actual understandability depends on the clarity of the classes, the interactions between them, and how well they match the developer's existing knowledge. In order to truly assess which design is easier to understand, an in-depth look at the specific implementation and structure of each design is required.

假设有人向你展示了一个系统的两个设计方案,设计 A 有 5 个类别,设计 B 有 9 个类别。对于这两种设计,你都认为类是合理的。你认为哪种设计更容易理解,为什么?

类的数量: 虽然设计 B 有更多的类,但如果这些类能使每个类的职责更清晰,并减少类之间的复杂性,那么设计 B 可能更容易理解。相比之下,设计 A 可能只有五个类,但如果它们之间的交互非常复杂,则可能难以理解。

有效性: 如果 "设计 B "中的每个类都有明确、单一的职责,并且与其他类的交互关系也很清晰,那么即使类再多,开发人员也更容易形成准确的心智模型。

效率: 如果 "设计 A "中的五个类功能相对集中,那么开发人员可能需要花费更多时间和精力来了解每个类的内部细节。而在设计 B 中,每个类可能更小、更集中,因此理解每个类是一项更快、更直接的任务。

结论 不能仅从类的数量来断定哪种设计更容易理解。实际的可理解性取决于类的清晰度、类之间的交互以及它们与开发人员现有知识的匹配程度。为了真正评估哪种设计更容易理解,需要深入研究每种设计的具体实现和结构。

可选回答1:

设计b具有更多数量的类,所以每个类更小
如果类更小,更可能匹配单一概念。如果一个类更能匹配在上下文模式或设计模式的单一概念,它越可理解。
如果类更小,他更可能只有一个职责和更具有凝聚力,所以可以花费更少的时间来阅读和理解类,这提高了效率,所以具有更高可理解性。

可选回答2:

B更容易理解:
1. 单一职责原则:划分更多的类意味着每个类具有更少的功能。在尝试理解程序时,可以减少每个类经由内化过程到Mind Model的时间。
2. 内聚性:A中的类之间可能有复杂的交互,B的类之间的关系可能相对简单。这同样使得内化过程大幅减少。
3. 抽象:更多的类可能意味着更高的抽象程度。这意味着在尝试理解程序时,无需理解一些类中的具体代码,可以从Knowledge Base中推测类的作用。