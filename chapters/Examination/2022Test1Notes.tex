\section{2022 Test 1 Notes}

对于每道题,请先写一遍题目中提到的定义。

\subsection{Q2}
\subsubsection{a}
软件维护的种类:corrective, adaptive, preventative, perfective。
增加一个特性是perfective。
\subsubsection{b 讨论可变性:}
可更改性是指如何在不引入缺陷或降低现有产品质量的前提下,有效且充分地更改产品系统。
4.3 中有很多地方需要修改,而 5.1 中的修改只有第 14、61、62 和 64 行(后三行都在一个方法 changePlayer() 中)。因此,仅仅修改 4.3 就需要更多的工作(\textbf{效率更低})。

由于要修改的地方较多,而每个地方都有可能出错(例如漏掉),因此要修改的地方越多,出错的可能性就越大(假设出错的可能性差不多--在本例中是一样的,因为在两种实现中要做的修改是一样的)。由于 4.3 需要改动的地方较多,在改动时出错的几率也较高,因此\textbf{有效性}较低。

提示:可变性与可理解性互相独立。本题中请不要讨论可理解性。

\subsection{Q3 类的大小/数量如何影响测试性}
比较:

小的类会比大的多(因为功能是一样的)。假设每个类都能提供相同水平的可观察性和可控制性,这意味着小的将比大的拥有更多的可观察性和可控制性,因此更容易测试。

事实上,小类可能比大类更具可观察性和可控性。类越大,需要为测试正确设置的状态就越多(Controllability),需要查看(Observability)的东西就越多,以便在测试后确认类中的对象是正确的。也就是说,需要花费更多精力来进行测试,这就不如小类的效率(Efficiency)高。

有单一责任(SRP)的类通常比没有单一责任的类小。

当一个类依赖于另一个类时(例如,通过调用另一个类中对象的方法),测试该类就需要额外的工作来设置另一个类中的对象。小类更有可能不依赖于其他类,因此所需的工作量会更少(Efficiency)。

\subsection{Q4 PCM}
\subsubsection{解释KB中的知识的特性}

这种知识优势必须存在(在知识库中),以便能够理解任何代码,但这种知识并不具体涉及所理解的代码。

String result = obj.someMethod(); 例如,对于这段代码:我们将使用 Java 语法知识来认识到 someMethod() 必须返回字符串(假设代码能编译)。知识库中的这些知识类似于 "当赋值运算符 (=) 右侧有一个方法调用,左侧有一个 java.lang.String 类型的变量时,该方法必须返回一个 String 值"。这不是专门关于代码的知识,因此不会出现在 MM 中。
MM 中的知识应该是这样的:"变量 obj 来自 A 类,A 类中的方法 someMethod() 返回一个描述对象当前状态的字符串"。

\subsubsection{解释陌生代码如何内化}
这段代码位于 ER 中。理解这段代码需要 Java 方面的知识,尤其是 = 左边和右边的含义。此外,还需要了解方法调用语法和方法调用的含义。这些知识与代码本身无关(见第(a)部分),因此必须放在知识库中。
如果您没有相关的 Java 知识,也就是说知识库中没有这些知识,那么您就需要获取这些知识。这就需要查阅有关 Java 的描述,这些描述可以在 ER 中找到。一旦掌握了这些知识,它们就会出现在知识库中。
一旦掌握了必要的 Java 知识,您就可以通过 AP 来解释代码(在 ER 中),从而确定代码的作用,并根据这些新信息更新 MM。

请注意,需要引用模型中的每一个组件。