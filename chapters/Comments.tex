\chapter{Comments}

\section{Good or Bad Comments}

代码异味(Code Smells)是指在代码中可以观察到的某些迹象,它们可能意味着背后存在更深层次的问题。这些不是错误——代码仍然可以运行——但它们暗示了设计上的不足,可能使代码难以维护和拓展。

Martin提到的“注释作为遮掩”的代码异味,强调了一个观点:如果代码需要大量注释来解释,那么它可能没有很好地表达自己的意图。优秀的代码应该是“自解释的”(self-explanatory),这意味着通过良好的命名和结构设计,代码应该尽量清晰和易于理解,而不是依赖注释。

但这并不是说所有的注释都是坏的。有时,注释是必要的,例如当解释某个复杂算法的原理或背景时,或者解释某些代码为什么要采取非常规的做法时。关键是区分何时应该使用注释,何时应该通过重构代码来提高其可读性和清晰度。

马丁以及很多其他人为什么对注释持负面态度?甚至好的注释也会带来额外的工作:
需要编写注释,这本身就是工作。
如果注释描述了代码的内容,并且代码发生了变化,那么注释至少需要被检查以确保它们仍然与代码保持一致(在最坏的情况下,需要修改注释)。
如果注释没有描述代码的内容,那么它们存在的意义是什么?
因此,注释最好能提供一些价值,通常是提高代码的可维护性。
它可以改善“同化过程”:不必阅读太多的代码,不必过多地思考代码可能的意义。

好的注释包括:法律、信息性、意图、澄清、后果、待办事项、扩展、API文档(例如Javadocs)。
不好的注释有:喃喃自语、冗余、误导、强制、日志、噪声、可怕的噪声、位置标记、属性、被注释掉的代码、HTML、非局部信息、过多的信息、不明显的联系、函数头。

有一件事你永远无法在代码中做到,那就是解释没有采取的路径。
当有多种方法可以做某事时,有人可能会看到该选项并想知道你为什么没有选择另一种选项——在注释中解释。

\section{Explain yourself in code}
选择一个描述性强且容易理解的变量名可以使代码更加高效。这是因为代码的阅读者不需要经常回到变量的声明部分,或者查看注释来理解变量的意义。
描述性的变量名可以减少对外部表示的依赖,增强代码的自描述性,从而使得代码理解的过程更加流畅和直观。
一个具有描述性的变量名可以帮助维护者快速建立和更新其心智模型,从而更高效地理解代码的功能和意图。

\section{注释的作用}
代码的开头有一个文档化的注释,解释了该方法的功能——给定一个数n,返回其对应的斐波那契数列中的数。此外,该注释还描述了方法的参数和返回值。
代码内部的注释提供了关于代码如何实现其功能的更多详细信息。

对于一些不那么直观的代码部分,注释可以帮助读者更快地理解代码的意图和功能,而无需深入分析代码的每一个部分。

考虑两种情况:一种是读者阅读注释来理解代码,另一种是读者通过仔细分析代码来重建代码的意图。
虽然两种情况都需要访问代码的外部表示(ER),但是没有注释的情况下,读者可能需要更多的时间或更深入地理解代码的各个部分,从而推断出代码的意图。因此,没有注释的情况下,理解代码的过程是不够高效的。

虽然注释很重要,但不必要的或冗余的注释会增加阅读和理解代码的工作量。例如,对于经验丰富的开发者来说,解释Fibonacci数列如何计算可能是多余的。
试图理解代码需要阅读这些注释以及它们解释的代码,如果注释是冗余的,那么这会使得理解过程更为复杂。

\section{Other comments practices}
\subsection{注释掉的代码}
有时,开发人员在修改代码时可能会注释掉某些部分,而不是完全删除它们,这通常是出于在将来可能再次需要这些代码的考虑。

分析:这种做法不推荐。一旦代码被注释掉,其他开发人员可能不敢删除它,因为他们不确定是否会再次需要这些代码。这最终会导致代码库中充斥着未使用的、注释掉的代码,增加了代码的复杂性和难以维护性。

\subsection{闭括号注释}
有时,开发人员在一个长方法或代码块的闭括号后面添加注释,以表示这个闭括号是结束哪个部分的。

分析:如果你觉得需要在闭括号后面添加注释,那么这通常意味着你的方法或代码块太长了。长代码块往往难以理解和维护。为了提高代码的可读性和维护性,应该考虑将代码块拆分成更小、更具有描述性的部分。
\subsection{位置标记}
位置标记通常用来在代码中标记特定的段落或部分,以方便其他开发人员快速导航。

分析:如果你觉得需要使用位置标记,那么这通常意味着你的类或方法太长了。如同闭括号注释的情况,过长的类或方法应该被拆分以提高代码的清晰度。
\subsection{TODO注释}
开发人员经常使用TODO注释来标记代码中尚未完成或需要进一步改进的部分。

分析:这些TODO注释有时被称为“自承认的技术债务”。它们表明代码在某些地方是不完整或不完美的。长期存在的TODO可能意味着项目中存在着悬而未决的问题或需要进一步的优化。
\subsection{注释作为沟通形式}
注释是开发人员之间关于代码意图、设计决策和功能的通信工具。

分析:由于注释是为人类(而不是机器)阅读的,因此它们应该是高质量的,能够清晰地传达必要的信息。质量差的注释可能会误导开发人员,导致他们对代码的错误理解。

\section{总结}
尽可能避免使用注释。支持理解的注释(可能还包括可维护性的其他方面)是可以接受的。