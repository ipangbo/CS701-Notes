\chapter{Code Smells}

\section{代码异味}

代码异味指的是代码中的某些特征,这些特征可能表明设计的质量存在问题。它不是明确的错误,而是代码中可能存在问题的迹象。

代码异味是基于启发式的,这意味着它们不是固定不变的规则。启发式原则提供了一个初步的方法或指引,但不保证总是正确。
因此,当我们在代码中发现异味时,不意味着一定存在问题,但确实表明我们需要更仔细地查看这部分代码,以确定是否真的存在设计问题。

这个概念源自Martin Fowler的书《Refactoring: Improving the design of existing code》。这本书是关于重构的,重构是一个软件工程的实践,目的是在不改变外部行为的前提下,改进代码的内部结构。
在这本书中,代码异味章节是由Martin Fowler与Kent Beck共同撰写的。Kent Beck是敏捷开发和测试驱动开发的先驱,与Martin Fowler一同在软件工程领域有着重要的影响。

\section{常见的代码异味}

\subsection{重复代码 (Duplicate Code)}这是最常见的代码异味。当在代码库中有重复的代码片段时,任何对这些代码的更改都需要在多个位置进行,增加了维护的复杂性。
\paragraph{修复方法}重构技术,如“提取方法”或“提取类”可以帮助消除重复代码。

\subsection{长方法 (Long Method)}当一个方法太长时,它通常做了太多的事情,违反了“单一职责原则”。
\paragraph{修复方法}可以通过“提取方法”重构来分解长方法,将其拆分成多个更小、更具描述性的方法。

\subsection{大类 (Large Class)}类似于长方法,一个类如果太大,可能承担了太多的职责。
\paragraph{修复方法}可以通过“提取类”重构来将大类分解为几个小类,每个类都有其明确的职责。

\subsection{长参数列表 (Long Parameter List)}方法有太多的参数,增加了理解和使用该方法的复杂性。
\paragraph{修复方法}使用参数对象或其他重构技术减少参数数量。

\subsection{投机性泛化 (Speculative Generality)}代码中包含的某些构造或功能,并不是因为当前的需求,而是基于未来可能的需求。
\paragraph{修复方法}删除未使用的代码或将其简化。

\subsection{数据类 (Data Class)}这些类只有字段(属性)但缺乏行为。这可能意味着代码的其他部分是在不恰当的地方处理与这些数据相关的逻辑。
\paragraph{修复方法}向数据类添加方法,或将相关逻辑移到适当的地方。

\subsection{注释作为除臭剂 (Comments as deodorant)}过度依赖注释来解释复杂或不清晰的代码。
\paragraph{修复方法}重构代码使其更具可读性,而不是仅依赖注释。

\subsection{临时字段 (Temporary Field)}这些字段只在某些特定条件下使用,可能会导致代码难以理解和出错。考虑使用更清晰的数据结构或对象,或重新考虑该逻辑。
\paragraph{修复方法}

\section{常见现象及对应的代码异味}
\subsection{Large Class}
“大类”是指在设计中,类的大小超过了它应有的范围,从而可能导致一些设计上的问题。但是,“大”是一个相对的概念,取决于上下文和项目的规模。

使用LOC(Lines of Code,代码行数)或其他绝对度量标准来定义类的大小可能不是一个好方法。因为对于一个10,000行代码的系统,一个100行的类可能不算大;但对于一个只有300行代码的系统,100行的类就显得很大了。

\subsubsection{大类可能带来的问题}

\paragraph{更高的变更风险(CR)}当类变得越大,它占据的代码总量也会增加。这意味着,在多种变更场景下,这个类都可能需要进行修改,从而增加了变更风险(Change Risk,CR)。这进一步降低了代码的可修改性(alterability)。

\paragraph{违反单一职责原则(SRP)}单一职责原则建议一个类只应有一个引起它变化的原因。当类过大时,它可能涉及多个功能或职责,从而违反了SRP。

\paragraph{更高的耦合度}大类可能与多个其他类交互,从而增加了耦合度。高耦合度意味着修改一个类可能会影响到其他类,这使得代码维护变得更困难。

\paragraph{可能的低内聚度}内聚度指的是一个类的各个部分(如方法或属性)如何紧密地关联在一起。当类很大时,它的各个部分可能并不都是为了同一目标而工作,导致低内聚度。这也可能使得类难以理解和维护。

\subsubsection{相关的代码异味}
\paragraph{长方法}方法越长,类就必须越大,从而导致大类。
\paragraph{长参数列表}参数越多,方法越长,从而导致长方法。































