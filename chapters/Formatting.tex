\chapter{Formatting}

\section{Readability}
阅读性是指读者能够理解书面文字的容易程度。在自然语言中,文本的阅读性取决于其内容(词汇和句法的复杂性)和其展现方式(例如字体大小、行高和行长度等排版方面)。单词的长度(字符数量)、单词的结构(音节)以及单词的熟悉度(之前是否见过这些单词)都会影响阅读性。

代码的可读性是理解性的一部分。在自然语言中,阅读和理解是不同的——对于源代码来说,我们可能需要区分这两者。

识别元素(例如循环体)或区分元素(例如一个变量与另一个变量)的难度越大,或确定哪些元素属于一起的难度越大,阅读的难度就越大。
如果代码难以阅读,那么理解它的有效性和/或效率就会降低。也就是说,较低的可读性导致较低的理解性。
但是,易于阅读并不意味着一定容易理解!

\section{Formatting}
格式化的规则通常影响可读性,从而也影响理解性。

\paragraph{可读性}是指代码中的元素可以被识别和区分的程度。

\subsection{Vertical Formatting}

垂直格式化关注的是代码文件的结构和内容的垂直布局。这包括文件的大小、内容在文件中的顺序、以及空行如何提供视觉提示。

\subsubsection{文件的大小}
根据Martin的观点,一个理想的代码文件大约应包含200行代码,但最多不超过500行。
这种推荐是基于可管理性和可读性的考虑。较小的文件更容易理解,也更容易维护。
\subsubsection{文件内的顺序}
新闻隐喻(“倒置金字塔”):这是指在新闻报道中,最重要的信息放在最前面,随后是较为次要的细节。在代码文件中,这意味着主要或核心的功能和概念应该在文件的顶部,而较为次要或细节性的内容应该放在下面。

密切相关的概念应该保持接近:这可以减少滚动,帮助开发者快速找到和某个概念相关的所有信息。

函数调用的顺序:在一些编程风格中,被调用的函数应该在调用它的函数的下面。这样做的目的是为了在读者阅读代码时,他们可以顺序地了解函数的调用流程。
\subsubsection{空白行}
空白行在代码中作为视觉提示,帮助区分不同的“概念”或代码块。例如,可以使用空白行来分隔函数、方法或不同的逻辑块。
这提高了代码的可读性,使代码更加结构化和整洁。

\subsection{Horizontal Formatting}

水平格式化关注的是代码在同一行内的布局和结构,如行宽、空白和缩进。

\subsubsection{行宽}
Martin推荐每行代码字符数为45。这考虑了阅读的舒适性,确保代码在一屏内可见,减少滚动和横向滚动。
\subsubsection{使用空白}
通过增加或减少空格,可以表示代码元素之间的关系有多紧密。例如,操作数和操作符之间的空格可以使表达式更易读。
\subsubsection{缩进}
缩进是代码层次结构的关键部分,用于表示块、循环和条件语句的起始和结束。它帮助读者理解代码的流程和逻辑结构。