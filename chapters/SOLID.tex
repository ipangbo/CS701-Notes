\chapter{SOLID原则}


\section{SOLID原则}
OLID原则在面向对象设计领域中被广泛认为是一套关键的设计原则。尽管大多数原则在Robert C. Martin之前已经以某种形式存在了,但正是他在2000年左右将这些原则组织起来并推广。

SOLID原则实际上是面向对象设计"正确"方法的结果。这意味着,如果设计者知道如何进行良好的面向对象设计,那么他们实际上可能不需要依赖这些原则。然而,对于初学者或者希望加强设计能力的开发者来说,SOLID原则是一个很好的起点。

\paragraph{Single Responsibility Principle(SRP)}一个类应该只有一个改变的原因。这个原则意味着一个类只应该负责一个特定的功能或职责。当需要修改类的某部分功能时,那么这个修改的原因应该是唯一的。这有助于确保类的职责清晰,并减少了修改类时对其他功能的潜在影响。
\paragraph{Open Closed Principle(OCP)}应该能够在不修改类的情况下扩展类的行为。软件实体应该对扩展开放,对修改封闭。这意味着应该能够添加新功能而无需修改现有代码。这通常通过使用接口、抽象类或者设计模式来实现。
\paragraph{Liskov Substitution Principle(LSP)}子类必须能够替换其父类。如果某个程序使用父类的对象,那么它也应该能够使用其子类的对象,而不会产生任何错误或异常。这确保了继承的正确性和多态性的一致性。
\paragraph{Interface Segregation Principle(ISP)}制作细粒度的接口,这些接口应该根据客户端的需要来定义。不应该强迫任何客户端实现他们不使用的接口或方法。接口应该小而专一,不应该把不相关的方法强加给实现它的类。
\paragraph{Dependency Inversion Principle(DIP)}依赖于抽象,而不是具体实现。高级模块不应该依赖于低级模块,它们都应该依赖于抽象。这有助于减少代码之间的耦合度,使得代码更易于重构和扩展。

\section{单一职责原则SRP}
一个类应该只有一个职责。这意味着当我们设计一个类时,它应该只服务于一个特定的功能或业务逻辑。此外,这一原则也强调类应该只有一个变动的原因。“变动”指的是修改类的源代码。通常,我们关心的变动类型是添加新功能,即在保持原有功能不变的情况下增加一些新的行为或特性。有一个更新的看法是将出于同一原因而变动的事物聚集在一起,而将出于不同原因变动的事物分开。这有助于在面对变化时,能够更灵活地进行调整,而不会对不需要改变的部分产生不必要的影响。

当一个类只专注于一个职责时,阅读和理解该类的功能会变得更加直接和简单。

虽然它被称为"单一职责原则",但一个可能更有意义的名称是"单一概念原则"。这是因为职责这个词可能有点模糊,而概念则更能明确地表示类应该代表一个单一、明确的想法或功能。

单一职责原则是面向对象设计中的核心原则之一,其主要目标是保持代码的模块化和可维护性。当我们设计系统时,应该努力确保每个类只负责一个特定的任务或功能,这样在未来需要变动或扩展时,我们可以更容易地进行修改,而不会对其他部分产生大量的连锁反应。

\subsection{职责驱动设计 (Responsibility-Driven Design)}
这种方法提供了一种新的、以职责为中心的方式来考虑和设计面向对象的软件。职责被定义为执行任务或知道某些信息的义务。这与常规的定义稍有不同,因为它关注于“做什么”和“知道什么”。

\subsubsection{职责驱动设计的步骤}
系统的职责是什么?这是确定系统应该完成的所有主要任务和它应该知道的所有信息的第一步。

哪些角色应该有这些职责?一旦确定了系统的职责,接下来的问题是确定哪些角色应该负责这些职责。角色可以视为抽象的实体或者是一组功能的集合。

哪些对象应该扮演给定的角色?在确定了各种角色及其职责后,接下来的步骤是确定哪些具体的对象应该扮演这些角色。对象是角色的实例,它们实际执行职责。

\subsubsection{职责与类的关系}
尽管名称中有“职责”这一词,但这并不意味着每个类只应有一个职责(至少根据这种职责的定义)。实际上,一个类可以有多个职责,这些职责共同协作以实现类的整体目的和功能。

职责驱动设计提供了一种以职责为中心的方法来设计面向对象的系统。它强调了确定系统的职责、确定应该扮演这些职责的角色,以及确定应该执行这些角色的具体对象的重要性。这种方法提供了一种更结构化、更组织化的方式来思考和设计软件,与传统的以数据或操作为中心的方法形成了对比。

\subsection{SRP与可变性}
当一个类或设计的改变原因更少时,它的可变性越高。换句话说,当一个类只有一个原因进行变更,它的设计更为健壮和灵活。

\subsubsection{案例}
设计A:有类 $A_1, A_2, ... , A_N$.

设计B:包含与设计A中相同的大多数类,但将$A_1$和$A_2$合并为一个新的类$B_1$。因此,设计B的总类数为$N - 1$。

\paragraph{变更$c_1$}当需要修改类$A_1$时,设计$A$的改变计数($C$)为1。然而,由于在设计$B$中,$A_1$已被合并到$B_1$,所以$B_1$也需要更改。因此,设计$A$和设计$B$的改变率($CR$)都为$\frac{1}{N}$。但由于设计$B$的N值较小,所以它的$CR$值较大。这意味着对于这种变更情境,设计$A$的可变性更高。
\paragraph{变更$c_2$}这种情况与$c_1$相似,只是变更的是$A_2$。根据同样的逻辑,设计$A$的可变性也是更高的。
\paragraph{变更$c_3$}当$A_1$和$A_2$都需要更改时,设计$B$只需更改一个类($B_1$),而设计$A$需要更改两个类。这意味着在这种情况下,设计$B$的可变性更高。

大多数情况下,设计$A$的可变性可能会优于设计$B$,但这取决于软件系统的整体生命周期中的具体变更需求。

\subsection{SRP与设计模式}
许多设计模式的元素都遵循单一职责原则(SRP)。这意味着这些元素被设计成只做一件事,并且它们应该只因一个原因发生变化。
\subsubsection{装饰器(Decorator)模式}
装饰器模式允许用户向单个对象添加新的功能,而不改变其结构。这种类型的设计模式属于结构模式,因为它涉及到多个类的组合来实现新的功能。
根据SRP,每个装饰器类应该有不同的变更原因。这意味着如果我们需要为一个对象添加新的功能或行为,我们可以简单地创建一个新的装饰器,而不是修改现有的代码。
\subsubsection{组合(Composite)模式}
组合模式用于当需要表示对象组成部分之间的整体/部分层次结构时。通过使用这种模式,我们可以使用递归结构创建一个树形结构,包含简单和复合对象。
根据SRP,每个叶节点和复合节点应该有不同的变更原因。这确保了我们可以独立地修改或扩展叶节点或复合节点,而不会影响到其他部分。
\subsubsection{命令(Command)模式}
命令模式用于将请求封装为一个对象,从而允许用户使用不同的请求、队列请求或记录请求日志,以及支持可撤销的操作。
根据SRP,每个具体的命令都应该有不同的变更原因。这意味着如果我们需要引入一个新的命令或操作,我们可以简单地添加一个新的命令类,而不是修改现有的命令类或逻辑。

\subsection{SPR总结}
一个职责"的定义是抽象的,它不提供具体、客观的操作来说明如何遵循这个原则,或者甚至如何识别这个原则没有被遵循。这种定义的抽象性可能会导致误解或模糊,因为不同的开发者可能会有不同的看法和理解关于"职责"是什么。

一个变化的原因"或"因为相同的原因而变化的事物应该放在一起"的定义可能更容易实施,因为这些定义提供了一个更明确的框架。这种思考方式是基于实际的变化案例,使得开发者可以更加具体地识别和组织代码。
当考虑变化的原因时,我们可以更直观地确定哪些部分的代码可能会因为同样的原因一起变化,从而确保它们在同一个类或模块中。这样,当某个特定的需求发生变化时,我们只需要修改一个地方,而不是多个分散的地方。

在实际开发过程中,开发者可能会发现,基于变化的原因来设计和组织代码是一个更实用和直观的方法。它为开发者提供了一个明确的路径,帮助他们决定如何最佳地组织和重构代码,以确保软件的可维护性和扩展性。

\section{开放封闭原则OCP}

开放封闭原则是面向对象设计中的一个基本原则,其核心思想由Bertrand Meyer提出,即"软件实体(如类、模块、函数等)应该对扩展开放,但对修改封闭"。这意味着在不修改现有代码的前提下,可以增加新的功能。

Meyer主要从"抽象父类"(也可以看作是接口)和它们的子类来探讨这一原则。子类可以通过继承父类来"扩展其行为",而不需要更改父类本身。
然而,这实际上并不改变父类的行为,而是提供了一个新的行为,即子类的行为。

Robert C. Martin在1996年的C++报告中(以及他后来的书中)重新解读了这一原则。
Martin的例子展示了这一原则真正的力量来源于多态(polymorphism)的明智使用,即通过动态分派来确定要执行的方法。
多态允许我们在运行时确定一个方法的具体实现,这使得软件更加灵活和可扩展。

开放封闭原则通常导致更多的抽象实体。这些抽象实体可以通过扩展或者通过将具体的实现作为参数提供(例如依赖注入)来具体化。
通过这种方式,系统可以在不修改现有代码的前提下,增加新的功能或者行为,从而提高软件的可维护性和扩展性。
\subsection{OCP与可变性}
OCP避免了修改已有的类,即减少了$C$。

\subsection{OCP与设计模式}
设计模式是为软件设计中反复出现的问题所制定的通用解决方案。开放封闭原则(OCP)是许多设计模式背后的核心思想,因为它鼓励创建稳定的、不可变的结构,同时提供灵活的扩展机制。以下是一些与OCP密切相关的设计模式示例:
\subsubsection{工厂方法 (Factory Method)}
工厂方法模式提供了一种在基类中创建对象的接口,但允许子类决定将实例化哪一个类。
通过工厂方法模式,不同的具体创建者类可以更改创建者组件的行为,以创建不同的产品。这意味着创建者的行为可以发生变化(开放)而不需要更改创建者本身的代码(封闭)
\subsubsection{模板方法 (Template Method)}
模板方法模式在一个方法中定义了一个算法的框架,将某些步骤延迟到子类中。
子类可以在不更改算法结构的情况下重新定义算法的某些特定步骤。在模板方法模式中,子类可以更改父类的行为(开放),而父类本身的结构和主要方法则不会发生变化(封闭)。例如,"BulkItem"可能是一个提供某些通用行为的父类,而其子类可以提供特定的实现细节,从而更改或增强其行为。

\subsection{OCP总结}

开放封闭原则的核心是为未来的变化做准备,同时避免对现有代码进行不必要的修改。这个原则鼓励我们在设计阶段就考虑到可能的变化,从而使代码更加稳定和可维护。

\paragraph{当我们更改某些内容时}
任何需要更改的部分都未封闭于更改。这意味着,如果某部分的代码需要频繁更改以满足新的需求或修复问题,那么这部分代码没有遵循开放封闭原则。开放封闭原则建议我们识别这些经常变化的部分,并通过适当的设计模式或方法对其进行重构,使其对于这些变化是开放的,但对其他不相关的变化是封闭的。
\paragraph{在设计某物时}思考上下文模式所需的变化。当考虑到这个模式的可能变化时,我们应该确保设计足够的灵活性,以适应这些变化,而不需要对现有代码进行大量修改。

\section{里氏替换原则LSP}

根据LSP,子类对象应当能够替代父类对象,并且不会产生任何错误或意外行为。简单地说,如果有一个地方可以使用父类的对象,那么我们同样应该可以在这里使用它的子类对象。在大多数面向对象的编程语言中,通常情况下,一个子类的对象可以合理地用在需要父类对象的地方,这是继承的基本属性之一。如果在子类中重写了某个方法,那么这个重写的版本必须与原始版本在语义上保持一致。换句话说,重写的方法应当“有意义”且不违背父类的预期行为。子类对象的行为应该与父类对象的行为相似,以致于使用父类对象的任何人或程序无法区分它们。

\subsection{LSP与可变性}
LSP是关于允许通过替换来改变行为的设计,而不要求使用父代(上下文)的代码发生变化,也就是减少了$C$。

\subsection{LSP与设计模式}
\textbf{任何依赖继承,或更准确地说,依赖多态性的设计模式(几乎所有的设计模式)都基于LSP。}

设计模式通常是针对常见问题的通用解决方案,很多设计模式都利用了对象之间的关系,特别是继承和多态性。
多态性允许对象以统一的方式对待,无论它们实际上属于哪个类。为了使多态性有效,子类的实例必须能够替代其父类的实例,这正是LSP的核心。
因此,为了保证使用继承的设计模式的正确性和有效性,它们必须符合LSP。

\subsection{LSP总结}
LSP提供了可行的标准来确定是否“正确”地使用了“继承”。当谈论“正确”地使用继承时,LSP定义了明确的标准。这不仅仅是从语法上正确地实现了继承,更重要的是子类的实例可以无缝地替代其父类的实例。

“正确”的含义不仅仅是代码没有错误,还意味着子类的对象能够在逻辑上和语义上替代父类的对象,而不会引入任何错误或异常行为。
对于LSP来说,“正确”意味着子类型(因此称为“替代”),即子类的对象应当可以替代父类的对象。

\subsection{LSP应用举例}

1. 什么是“替换”

有两个类NewZealandMoney和AussieMoney,它们都实现了Money接口。
Client类有一个静态方法whatDoIHave,它期望一个Money对象作为参数。
当传入NewZealandMoney对象时,该方法可以正常工作,因为NewZealandMoney实现了Money接口。
这说明了LSP的关键点:一个基类或接口的引用可以被其子类的对象替换,并且不会导致程序出错。

2. 替换不是有用的地方

第一个例子中,NewZealandMoney对象被赋给了一个Money类型的引用,这是允许的。
第二个例子中,NewZealandMoney对象被赋给了一个Object类型的引用。尽管这在Java中是允许的,但它不能调用formalString方法,因为Object类没有这个方法。
第三个例子尝试将Cat对象传递给whatDoIHave方法,这将导致错误。
这些例子说明,仅仅因为一个对象可以被赋给另一个类型的引用,并不意味着它们之间存在有意义的“替换”。

3. 代替继承

介绍了使用组合代替继承的方法。NewZealandMoney和AussieMoney类现在不再继承DollarMoney,而是包含一个DollarMoney类型的私有成员。
通过“转发”或“委托”方法,这些类提供了与Money接口兼容的实现。
这种方法是基于“优先考虑对象组合而不是类继承”原则,这是“四人帮”在其“设计模式”书中提到的。

结论:

使用继承时,应确保子类可以在任何需要基类的地方进行替换,这是LSP的核心。
当不需要替换或当继承可能导致问题时,可以考虑使用组合代替继承。


\section{接口隔离原则ISP}
ISP建议创建细粒度的接口(fine-grainedinterfaces),这些接口应该是为特定的客户端量身定制的。这意味着每个接口都应该专门为一组特定的任务或操作进行设计,而不是为多个不相关的任务定义一个大型、全面的接口。

客户端不应该被迫依赖于它们不使用的方法。如果客户端被迫实现它不需要的接口方法,那么当这些方法发生变化时,即使客户端不使用它们,客户端也可能需要进行更改。这会导致不必要的复杂性和可能的错误。

\subsubsection{SRP与ISP的关系}
单一职责原则(Single Responsibility Principle, SRP)通常意味着一个类只应该有一个改变的理由。遵循SRP的类的客户端原则上应该需要其提供的所有方法。然而,有时这可能不太实际。
例如,一个“银行账户”可能有多种不同的概念(例如储蓄账户、支票账户、定期存款),而某些客户端可能只关心这些账户的一般类别而不是具体类型。
此外,遗留代码或特定的实现需求可能意味着在同一个类中有不同的职责。在这种情况下,ISP可以帮助确保客户端只依赖于它们真正关心的那部分功能。

当有多个独特的概念属于一个更通用的类别时,ISP尤为有用。客户端可能只关心总体类别,而不是具体的子类别。
遗留代码可能不完全符合现代的设计原则,但仍然需要与新的系统集成。ISP可以帮助减少新系统与旧代码之间的紧密耦合。
当实现需求导致一个类有多个职责时,ISP可以确保这些职责被适当地分隔到不同的接口中。

\subsection{ISP与可变性}

一个依赖于它不使用的方法的客户端可能会在该方法发生变化时进行更改--它对该方法产生了不必要的依赖。消除这种依赖会降低客户端需要更改的可能性,也就是减少$C$。

A client that depends on a method it does not use may then have to change when that method changes—it has an unnecessary dependency on that method. Removing the dependency reduces the likelihood the client needs to change, so reducing $C$.
\subsection{ISP与设计模式}
设计模式常常定义小的接口来实现特定的模式。然而,实际使用中的类可能会有更大的接口。这意味着这些类除了用于实现设计模式的方法外,可能还有其他方法。

\subsubsection{Observer模式}
在 Observer 模式中,存在一种“观察者”和“被观察者”的关系。当被观察者的状态发生变化时,所有注册的观察者都会收到通知。
观察者类(Observer)通常需要实现一个名为 update() 的方法,用于接收来自被观察者的状态变更通知。
尽管 update() 方法是观察者模式中观察者需要实现的主要方法,但实际的观察者类可能会有许多其他方法,因此它的接口可能会更大。
\subsubsection{Command模式}
Command 模式旨在将请求封装为一个对象,从而允许用户使用不同的请求、队列请求或记录请求日志,以及支持可撤销的操作。
Command 对象通常只有一个 execute() 方法,该方法定义了命令的具体行为。
与 Observer 模式不同,Command 模式的对象通常仅有一个核心方法,因此其接口相对较小。

\subsection{ISP应用举例}

客户端可以需要 Artist 和 Date 对象,但如果他们只需要比较这些对象(例如,在排序算法中),那么他们不需要依赖于这些对象确实是 Artist 或 Date。
Comparable 接口为那些只关心比较对象的客户端提供了一个精细化的接口。也就是说,客户端可以简单地使用 Comparable 接口来进行排序,而不必知道背后的实际对象类型是什么。

\subsection{ISP总结}
原则上,可以确定一个类实际使用的方法集合,并创建一个完全匹配该集合的接口。这意味着,如果一个类的某些方法只被某些客户端使用,那么这些方法应该被放在一个单独的接口中,而不是放在一个大而全的接口中。尽管接口隔离原则鼓励创建细粒度的接口,但过度执行此原则可能导致大量的小接口。这些小接口可能如此抽象,以至于它们失去了任何有用的意义。


\section{依赖反转原则DIP}
高层次的策略性代码不应该直接依赖于低层次的实现性代码。相反,它们都应该依赖于共同的抽象。

Hollywood principle — ``don't call us, we'll call you":好莱坞原则与DIP紧密相关,它强调了控制的反转。在传统的程序设计中,主程序调用方法或子例程。但在依赖反转和好莱坞原则中,控制反转了,即框架或父例程决定何时调用子例程或子模块。

控制反转(IoC)是一个更广泛的概念,它描述了使用接口或抽象来实现模块之间的灵活关系,从而减少它们之间的直接依赖。DIP是IoC的一种具体实现。

\subsection{DIP与可变性}
A class that depends on concrete types is more likely to need to change if any of the concrete types change than if it depends on abstract types, so reduces $C$.

与依赖于抽象类型相比,依赖于具体类型的描述在具体类型发生变化时更有可能发生变化,也就是减少了$C$。

\subsection{DIP与设计模式}

\subsubsection{依赖注入}
Dependency Injection(依赖注入)是实现DIP的一种技术。它提供了一种机制,使得客户端模块不直接依赖于具体的服务实现,而是依赖于服务的抽象。在运行时,具体的服务实现(或称为具体的依赖)被“注入”到客户端模块中。
这意味着客户端模块不需要知道或关心它使用的服务是怎么实现的,只需要知道服务的接口或抽象。这增强了模块间的解耦,并提高了代码的灵活性和可维护性。

\subsubsection{策略模式}

Strategy Pattern(策略模式)是一种行为设计模式,它定义了一系列算法,并将每一种算法封装起来,使它们可以互相替换。这样,算法可以独立于使用它的客户端变化。
在策略模式中,“Context”是使用某种策略或算法的客户端。但它并不直接依赖于具体的策略实现,而是依赖于策略的抽象。这正是DIP的核心思想。
通过将策略定义为一个接口或抽象类,并让具体的策略实现这个接口或继承这个抽象类,策略模式允许我们轻松地替换或变更算法,而不需要修改使用它的客户端代码。

\subsection{DIP总结}

这个原则提供了实际可行的建议或操作。在编写或设计软件时,应当优先考虑使用抽象而不是具体实现。这意味着在定义依赖关系时,尽量依赖于接口或抽象类,而不是具体的类。
这种方法提高了代码的可维护性和可扩展性,因为更改或替换某个具体实现不会影响到依赖于抽象的其他部分。

虽然DIP鼓励依赖于抽象,但对于基本数据类型(如int、float、string等,primitive types)的处理可能会产生一些不确定性。基本数据类型通常被视为具体的,并且它们本身就是语言的一部分,所以很难为它们提供抽象。
但在某些情况下,为了提高灵活性和代码的可读性,可以通过使用特定的数据类型或类来封装基本数据类型。
在使用第三方库或外部软件时,可能没有提供足够的抽象。这是因为库或软件可能已经被定义为具体的实现,而不是抽象的接口。
在这种情况下,如果需要,可以考虑在自己的应用程序中为这些第三方库提供一层抽象,以确保自己的代码与这些库之间解耦。


\section{总结}

由于某些原则具有主观性,它们可能难以遵循。例如,单一职责原则(SRP)定义可能会因项目和团队而异。

有些原则,如依赖反转原则(DIP)和里氏替换原则(LSP),其应用效果是可以量化的。

SOLID原则之间并不是独立的。例如,只有当遵循了SRP的时候,DIP才能很好地工作;只有当所有子类型都使用LSP的时候,注入的类型才能正常工作;开闭原则(OCP)在遵循SRP的情况下才能获得最大的益处;接口隔离原则(ISP)实际上是基于LSP的。

SOLID原则都在一定程度上利用了多态性。

尽管SOLID原则为设计带来了许多好处,但始终需要权衡成本和效益。例如,遵循SRP可能需要添加更多的类,这会带来创建、测试和维护的成本。开发者需要考虑这些成本,并确定遵循该原则的好处是什么。





