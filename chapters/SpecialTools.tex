\chapter{特别专题:评判可维护性的工具}

\section{评判可理解性的工具}

即使代码格式化得很好,变量和方法的命名也很贴切,但这并不能有效地帮助理解一个拥有上百万行(或者甚至上万行)的程序。为了更好地理解庞大的代码,一个通用的方法是创建工具,这些工具可以提供关注代码“重要”方面的视图。这意味着通过某种方式过滤或聚焦于代码的关键部分,而不是尝试一次性理解整个代码。

\subsubsection{具体方法}
\begin{itemize}
	\item 程序分析工具 (Program Analysis Tools): 这些工具可以自动分析代码,找出其中的模式、依赖关系和可能的问题点。例如,静态代码分析工具可以检测潜在的错误、安全性问题或不良的编码实践。
	\item 软件可视化工具 (Software Visualisation Tools): 这些工具将代码或其结构转化为图形或图像,以便于人类理解。例如,依赖关系图可以显示不同模块之间的依赖,而流程图可以显示程序的执行流程。
	\item 编程语言 (Programming Languages): 选择正确的编程语言或者利用语言的特定功能也可以帮助更好地理解程序。某些编程语言可能通过其设计或特性,如强类型、函数式编程或面向对象的特性,来促进更清晰、更模块化的代码。
\end{itemize}


\subsubsection{程序分析的重要性}
工具需要分析代码以确定感兴趣的属性。这种分析可以揭示代码的各种特征,从而帮助开发者更好地理解和优化代码。

\subsubsection{代码分析的步骤}
\begin{itemize}
	\item 解析代码: 工具首先需要读取并解析代码。
	\item 创建内部表示: 通常,工具会创建一个代码的内部表示,称为"抽象语法树"(AST)。
	\item 查找特征: 内部表示然后被检查,以找到感兴趣的特征。
\end{itemize}


\subsubsection{软件指标}
程序分析工具通常提供来自"软件指标"的测量,这些测量可以帮助评估代码的质量、效率和其他关键特性。

\subsubsection{实例分析}
以下是一些可能的研究或分析主题,其中包括了如何检测代码中的不良现象、开发者在命名、编码和决策中的实践等。
\begin{itemize}
	\item 代码的“坏味道”: 如何检测代码中可能存在的设计或实现问题。
	\item 命名规范的忽视: 开发者有多频繁地忽略了命名标识符的指南。
	\item 非有意义的命名: 开发者使用不具有明确意义的命名的频率。
	\item 函数/方法的参数数量: 开发者在他们创建的函数/方法中有多少参数。
	\item 记录不良决策: 开发者有多频繁地在代码中记录他们所做的不良决策(被称为"自承认的技术债务")。
	\item 代码量的测量: “代码行数”是否提供与“方法数量”或“字段数量”不同的信息。
	\item 关注点的分离度量: 如何确定代码中"关注点分离"的程度。
\end{itemize}



\section{可视化工具}
软件可视化是软件工程领域中的一个大研究领域。例如,IEEE的VISSOFT工作会议专门研究软件可视化,这显示了这一主题的重要性和相关性。一幅画胜过千言万语" — 这句话表达了软件可视化的核心思想,即使用图片来有效地展示软件中的感兴趣方面。通过图像,我们可以更直观、更快速地理解复杂的概念或数据。

软件可视化强调了多个领域的结合,包括排版、平面设计、动画、电影摄影,以及现代的人机交互和计算机图形技术。这些结合在一起,目的是促进人们对计算机软件的理解和有效使用。











